\chapter{Top-\emph{k} Joining Language Models}
\label{ch:topkjoin}

\todo[inline]{Add algorithm that only uses sorted access.}

\lstset{
  language=SQL,
  %basicstyle=\ttfamily,
  emph={arg,Pattern,Pattern1,PatternN,P_LM},
  emphstyle=\textit,
  deletekeywords={count} % Used as column name, don't want to escape it though.
}

This chapter explains how to efficiently compute arg-max-queries of language
model $\Prob[\text{LM}]$ using top-$k$ joining techniques.

The task is to efficiently compute the following expression:
\begin{equation}
  \Argmax_{t \in T} \Prob[\text{LM}]{t}{h}
\end{equation}
Where $t$ is a token in the set of all tokens $T$, $\Prob[\text{LM}]$ is a
language model, that can be expressed as a weighted sum (as described in
\cref{ch:weightedsum}), and $h$ is the conditional history, a sequence
of tokens.

Given a language model $\Prob[\text{LM}]$ as a scoring function, if we
imagine our input data forming database relations, we can express our
arg-max-query as a SQL-\texttt{SELECT}-Request, as given in \cref{lst:topksql}.

\begin{lstlisting}[
  label = {lst:topksql},
  float,
  caption = {\todo[inline]{Example SQL-Request Caption}},
]
  SELECT
    Token.token AS arg,
    $\Prob[\text{LM}]$(Pattern1.count, ..., PatternN.count) AS score
  FROM Token
  LEFT OUTER JOIN Pattern1
    ON history(Pattern1) + arg = Pattern1.sequence
  ...
  LEFT OUTER JOIN PatternN
    ON history(PatternN) + arg = PatternN.sequence
  ORDER BY score DESC
  LIMIT k;
\end{lstlisting}

\todo[inline]{Where \inlinecode{history} and plus is concat.}
\todo[inline]{Explain \inlinecode{LEFT OUTER JOIN}.}

Our data model is thus: One input relation \inlinecode{Token} that stores
all possible tokens \inlinecode{arg}, over which we want to compute:
\begin{equation}
  \Argmax_{\text{\inlinecode{arg}} \, \in \, \text{\inlinecode{Token}}}
    \Prob[\text{LM}]{\text{\inlinecode{arg}}}{\text{\inlinecode{history}}}
\end{equation}
Additionally we have $n$ input relations
\inlinecode{Pattern1} to \inlinecode{PatternN} that contain pairs
of \inlinecode{sequence}s and the sequence's occurrence \inlinecode{count} as
columns.
Each \inlinecode{Pattern} relation contains exactly these sequences that fit the
relations pattern.

For an example database state and fully expanded SQL-Request see
\cref{app:sqlenv}.

A na\"{\i}ve method to solve this query would
\begin{inparaenum}[(1)]
  \item produce all possible object combinations that satisfy the join
    condition,
  \item order results by scoring function $\Prob[\text{LM}]$, and
  \item discard non top-$k$ objects (those that do not have the $k$ largest
    scores).
\end{inparaenum}

Considering that in reality only small values of $k$ are queried, this approach
can be deemed as rather wasteful, as large numbers of objects would have to be
discarded.
Optimally we would like to cherry-pick the top-$k$ join results and only
score and order those.
Relying on the fact that we are dealing with a \emph{monotone} scoring function
we approximate this by sorting our input relations by their counts.
\textcite{Ilyas2008} survey a wide number of techniques to produces top-$k$ results
with a monotone scoring function from sorted relations.

Among other things, top-$k$ joining techniques differ on the type of join
conditions they can be applied to.
One particularly optimizable join operation is the \emph{equi-join}:
the join that checks for equality over a unique key attribute present in all
relations.
Equi-joins have the advantage that it is possible to form equivalence classes
over an object's key attribute to determine which join results an object is
part of.
Using this, the \emph{hash join} algorithm can efficiently generate join
results.

It is possible to view our join operations as an instance of equi-join:
By filtering each input relation \inlinecode{Pattern} to only contain
objects whose sequences begin with the requested history, objects'
sequences only differ on the last token (our argument token \inlinecode{arg}).
Our join condition thus reduces to
\inlinecode{arg = lastToken(Pattern.sequence)}.

\todo[inline]{Read \textit{``2013 BlanasPatel Memory Footprint Matters Efficient
Equi-Join Algorithms fro Main Memory Data Processing.pdf''} for possibly better
join algorithm.}

The technique for obtaining these both sorted and filtered relations is
explained in \cref{sec:sorted-and-filtered-access}.

\todo[inline]{Do we want discussion here rather than explanation?}

The algorithm used to produce top-$k$ results is presented in
\cref{sec:algorithm}.

% ------------------------------------------------------------------------------
\section{Remarks}

% ------------------------------------------------------------------------------
\section{Sorted and Filtered Access}
\label{sec:sorted-and-filtered-access}

As explained we require a data structure for efficiently retrieving all
sequences whose first to second to last words match a requested string $h$.
Additionally we want to specify that the last word has to contain $p$ as a
prefix, in order to support efficient Next-Keystroke-Savings $\NKSS$
computation.
\mbox{$\Set{\, w_1^n \: \Given \: w_1^{n-1} = h \; \land \; b \:\text{is prefix of}\: w_n \,}$}
which is equivalent to
\mbox{$\Set{\, w_1^n \: \Given \: \,}$}
\begin{equation*}
  \underbracket{\enskip w_1 \qquad w_2 \qquad \ldots \qquad w_{n-1} \enskip}
    _\text{matches $h$}
  \quad
  \underbracket{\enskip w_n \enskip}_\text{starts with $b$}
\end{equation*}

NKSS requires Top-\emph{k} Completion.

Use Completion-Trie \parencite{HsuOttaviano2013}.

\todo[inline]{Explain Completion-Trie here?}
\todo[inline]{Can we save all patterns in one trie?}

% ------------------------------------------------------------------------------
\section{Algorithm}
\label{sec:algorithm}

\subsection{Requirements}

\begin{enumerate}
  \item Dataset remains the same.
  \item Scoring function changes on each query.
  \item Join condition changes on each query!
\end{enumerate}

\subsection{Algorithm}

There are several techniques for indexing {Rank Join Indices \mbref{Tsaparas et
al 2003}, Onion Indices \mbref{Chang et al 2000}) which we don't use as history
changes to often.

Ranked Join Indicies\mbref{Tsaparas et al 2003})

Algorithms that precompute stuff may be general speed up, but not in our case
since we never expect the same thing to be queried again.

TA \parencite{Fagin2001}.

Rank-Join \parencite{Ilyas2004}.

Evaluate following algorithms:

\begin{enumerate}
  \item Traditional Rank-Join that generates all possible join combinations
    after each input.
  \item Improved version that uses random access for early tuple completion.
\end{enumerate}

Use observation that if a pattern is seen, it's child patterns' will be seen as
well.

Clever: How to find T as max of $(p_1^{max}, p_2^{last})$ and
$(p_1^{last}, p_2^{max})$.

Clever: Priority-Queue with constant contains and update.

\section{Beam search}

Beam search is a path finding algorithm that always expands the most promising
states, and terminates when a solution was found.

This however is not directly applicable to our problem, as finding a solution
is typically very easy (get top node from sorted pattern counts and take it as
solution).

How to interpret: expanding the most promising state?
\begin{enumerate}
  \item Peek sorted counts, list with highest count is most promising.
  \item Now how to determine if we want to do random access to missing pattern
    counts or rather fetch next sorted count.
  \item Could do beam width many sorted accesses, and complete missing patterns
    with random acccess and call it a day.
\end{enumerate}

For random access the cost of an edge is only known, once the edge is walked.
For sorted access we can peek.
