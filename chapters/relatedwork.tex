\chapter{Related Work}
\label{ch:relatedwork}

% ------------------------------------------------------------------------------
\section{Top-\emph{k} Joining Techniques}

\textcite{Ilyas2008} survey a wide range of different top-$k$ processing
techniques.
One way to distinguish top-$k$ techniques is data access.
Nearly all algorithms necessitate some form of \emph{sorted access} to data
sources.
Some techniques additionally require \emph{random access}.

Among others, two fundamental algorithms were established by
\textcite{Fagin2001}: the \emph{threshold algorithm} (sorted and random access)
and the \emph{no-random-access algorithm} (only sorted access).
The implementation for these two algorithms will be described in
\cref{ch:topkjoin}, and we will evaluate them in \cref{ch:evaluation}.

\begin{draft}
Mention instance optimiallity, and else Fagin proved.
\end{draft}


\begin{draft}
Mention \textcite{Guentzer2000} and \textcite{Guentzer2001} improvements to TA
and NRA.
\end{draft}

\begin{draft}
Mention other more optimized techniques such as onion indices or J* or
Rank-Join.
How do I explain these can not be used in this chapter?
\end{draft}
