\chapter{Related Work}

\begin{draft}
Commonly to solve the described problem Viterbi or beam search algorithms are
used \mbref{JurafskyMartin2009}.
\end{draft}

\todo[inline]{Is there some work on expressing language models as weighted sums?}

% ------------------------------------------------------------------------------
\section{State-of-the-art Language Models}

\begin{draft}
This section summarizes the state-of-the-art language models considered in this
work.
And their origins?
\end{draft}

The currently most commonly used \parencite{JurafskyMartin2009,Chelba2013}
technique for estimating language models is \emph{Modified Kneser-Ney Smoothing}
by \textcite{ChenGoodman1996,ChenGoodman1998,ChenGoodman1999}, based on
\emph{Kneser-Ney Smoothing} by \textcite{KneserNey1995}.
Very recently \textcite{Pickhardt2014} provided a generalization of this technique,
the \emph{Generalized Language Model}.

\begin{draft}
Justify selection of MKN and GLM over recently popular LMs from
\parencite{Chelba2013}.

Mention \emph{Nerual Network Language Models} \parencite{Bengio2003,Mikolov2012}
(do we not use them because they can not be represented as weighted sums).
\end{draft}

An in-depth summary of Modified Kneser-Ney Smoothing and the Generalized
Language Model is given in \cref{ch:review-lm}.

% ------------------------------------------------------------------------------
\section{Top-\emph{k} Joining Techniques}

\Lukas[inline]{I would much rather summarize top-$k$ joining techniques in
\cref{ch:top-k-joining} as the task is clearly defined there, and we can have
more in-depth discussion about it.}

\begin{draft}
Reference to \cref{ch:top-k-joining}.
\end{draft}
