\chapter{Introduction}
\label{ch:introduction}

\todo[inline]{
  Rewrite introduction with new title in mind:
  NWP is important.
  NWP is hard and slow.
  NWP quality is measured with NKSS.
  NWP and NKSS form prefix queries.
  I make fast prefix queries.
}

\begin{draft}
\textcite{Bickel2005} actually does trivial argmax and calculates probabilities
for all words in vocabulary repeatedly.
\end{draft}

The \emph{noisy channel model} is a commonly applied concept in natural language
processing.
As first described by \textcite{Shannon1948}, it consists of a transmitter that
tries to pass a message over a channel to a receiver.
The channel is called noisy, because it may distort the message in any way.
The task is then to reconstruct the original message, from the recieved,
distorted one.

This metaphor has many applications, such as spelling correction
\parencite{JurafskyMartin2009,Manning2008,Kernighan1990,Mays1991},
%          \------------- non-word-errors ------------/    \- real-word-erros
word prediction \parencite{Bickel2005}, part-of-speech tagging
\parencite{Church1988}, machine translation \parencite{Brown1990}, speech
recognition or text compression.
\todo{Newer citations!}
% JurafskyMartin2009: Chapter 5: Spelling Correction
%                     Chapter 9: Speech Recognition!

This thesis will present a novel approach to solve noisy channel type queries.
The basic idea is to formulate the query as the argmax of a monotone scoring
function, and then to find the solution using \emph{top-$k$ joining techniques}.
This approach will exemplary be showcased by applying it to the problem of
\emph{next word prediction}.

% ------------------------------------------------------------------------------
\section{Noisy Channel Model}

\Rene[noline]{Probability space unclear. Most of the time not exactly stated
when talking about Noisy Channel Theorem. Maybe not bring it up in bachelor
thesis.}
Formalized, the task of the noisy channel model is to find the dispatched
message~$m$, which in our context is the sequence of words $s$ in a
langauge~$\Language$ that is most likely to be the source of the received,
distorted observation~$o$:
\begin{align}
  \label{eq:noisychannel}
  m &= \Argmax{s \in \Language} \Prob{s}{o} \\
  \intertext{Commonly this noisy channel equation is rewritten using Bayes'
    theorem to:}
  \label{eq:noisychannelbayes}
  \begin{split}
    m &= \Argmax{s \in \Language} \frac{\Prob{o}{s} \cdot \Prob{s}}{\Prob{o}}
       = \Argmax{s \in \Language} \Prob{o}{s} \cdot \Prob{s}
  \end{split}
\end{align}
With the observation likelihood~$\Prob{o}{s}$ and the prior
probability~$\Prob{s}$.
Omitting the denominator~$\Prob{o}$ is valid because it is a constant that does
not depend on the argument~$s$.
Most commonly the observation likelihood is some domain specific probability
distribution $\Prob[\text{Domain}]{o}{s}$.
For the prior probability, \emph{statistical language model} distributions
$\Prob[\text{LM}]{s}$ are used.
%They assign a probability $\Prob[\text{LM}]{s}$ to a sequence of words
%$s$ by means of a probability distribution, created through statistical analysis
%of text corpora.
Thus the noisy channel model is a natural combintion of domain knowledge with
language information.

For example, to translate French into English, \textcite{Brown1990} defined a
probability distribution $\Prob[\text{Translate}]{e}{f}$, that would give the
probability for each English sentence $e$ to be the result of translating
the French sentence $f$ into English.
To obtain the most likely translation, the $e$ that maximizes that probability
has to be found.
With applying Bayes' theorem they combine their distribution with language
models and arrive at
$\Argmax{e} \Prob[\text{Translate}]{f}{e} \cdot \Prob[\text{LM}]{e}$, which is
just \cref{eq:noisychannelbayes}.

% ------------------------------------------------------------------------------
\section{Next Word Prediction}

In this thesis the problem of \emph{next word prediction} will be researched.
For a sequence of words $w_1 \ldots w_n = w_1^{n}$, the task is to find the last word
$w_n$, given only the first $n-1$ words $w_1^{n-1}$.
This is done by finding the $w$ that maximizes the probability to
occur after $w_1^{n-1}$ \parencite{Bickel2005}:
\begin{equation}
  \label{eq:nwp}
  w_n = \Argmax{w \in \Vocab} \Prob{w}{w_1^{n-1}}
\end{equation}
\begin{draft}
\begin{equation}
  \label{eq:prefixquery}
  \NWP[p][k]{h} =
    \Argmax*[k]{\mathclap{\substack{w \in \Vocab,\\p \: \text{prefix of} \: w}}}
      \Prob{w}{h}
\end{equation}
\begin{subequations}
  \begin{align}
    \Argmax [1]{x} f(x) &= \big\{ x \,\Given\, \forall y : f(y) \leq f(x) \big\} \\
    \Argmax [k]{x} f(x) &= \big\{ x \,\Given\, \forall y \notin \Argmax*[k-1]{z} f(z) : f(y) \leq f(x) \big\} \\
    \Argmax*[k]{x} f(x) &= \bigcup_{{1 \leq i \leq k}} \Argmax[i]{x} f(x)
  \end{align}
\end{subequations}
\todo{Notation for $w\ \text{prefix of}\ p$?}
\todo{How do we define NWP, in case there is not one clear argmax?}
We call $h$ the history and $p$ the prefix.
\end{draft}
With vocabulary $\Vocab$, the set of all words.

We can see that next word prediction is an instance of the noisy channel model,
as specified in \cref{eq:noisychannel}.

As there is no domain knowledge to apply, one way to solve next word prediction
is to directly calculate probabilities $\Prob{w}{w_1^{n-1}}$ by using
\emph{statistical language models} \parencite{Bickel2005}.

Next word prediction techniques are often evaluated by measuring \emph{next
keystroke savings} \parencite{Bickel2005}\todo{More citations!}.
\begin{draft}Next Keystroke savings are: What actually?\end{draft}
As such, we would like our implementation to also support the computation of
next keystroke savings.
%A metric closely related to next word prediction is \emph{next keystroke
%savings}.
%\Lukas[inline]{How do we define NKSS? How do I justify it's use in this thesis?
%I need NKSS to justify my choice of completion-tries as data structure :).}

% ------------------------------------------------------------------------------
\section{Top-\emph{k} Join Queries}

Top-$k$ queries are queries whose results are limited to only the $k$ most
important ones.
These top-$k$ results are identified by scoring all answers with a
\emph{scoring function}, and selecting the $k$ highest scoring.
\emph{Top-$k$ join queries} are these top-$k$ queries that produce
their results trough combination of multiple data sources \parencite{Ilyas2008}.

One example might be the query to a web search engine.
Search results are ranked by relevance, as computed by several criteria
(matching keywords, links from other sites, popularity, \ldots).
Users generally require more than just the one highest ranking search result, as
it may not be exactly what they are they were looking for.
On the other hand it would be infeasible to relevance-rank all sites in the
search engine's database.
Many search engines solve this by only displaying the top 10 sites, deemed most
relevant to the user's query, and querying more if required.

\begin{draft}
Top-$k$ joining techniques have widespread use and are of importance in
many applications \noref.
\end{draft}

As can be seen by the example, the na\"{\i}ve way to solve top-$k$ join
queries, i.e.\ to aggregate all possible join results, sort all of them by
score, and then to discard non top-$k$ results, is in most cases far to
slow to be of any use.
There are of plethora of different top-$k$ processing techniques, but most
of them achieve efficiency by requiring \emph{monotone scoring functions} and
then sorting data sources on each predicate of that function
\parencite{Ilyas2008}.

% ------------------------------------------------------------------------------
\section{Statistical Language Models}

\emph{Statistical language models} assign a probability $\Prob[\text{LM}]{s}$ to
a sequence of words $s = w_1^n = w_1 \ldots w_n$ by means of a probability
distribution, created through statistical analysis of text corpora.

A major problem in creating statistical language models from text corpora, is
that even for very large corpora, most possible word sequences in a language
will not be observed in training \noref.
A common approximation is to make a $n$th-order \emph{markov assumption},
i.e.\ to assume that the probability of the $n$th word $w_n$ only depends on
the $n\!-\!1$~previous words $w_1^{n-1}$.
\todo{Problem with indices: They suggest that there are no words before $w_1$.}
Thus only sequences of length $n$ have to be considered.
The resulting language models are called \emph{$n$-gram models}.
State-of-the-art language modeling uses $n$-grams with $n$ as large as $5$
\parencite{JurafskyMartin2009,Goodman2001,Stolcke2000}.

\begin{draft}
Calculating noisy channel type queries (\cref{eq:noisychannel}) is a very time
expensive operation, as for each.
\Cref{eq:noisychannel} is hard to calculate for such large $n$.
So we want to optimize that.
\end{draft}

Most language model probability formulas are given in recursive form, that only
implicitly depend on the occurrence count of sequences in the corpus \noref.
If these probabilities can be expressed as monotone scoring functions of
occurrence counts, we can apply top-$k$ joining techniques in order to solve
noisy channel type queries.

% ------------------------------------------------------------------------------
\section{Thesis' contribution}

\todo[noline]{Wouldn't this be better as an abstract? How do I introduce here
instead?}
The concern of this thesis is solving noisy channel type queries that are based
on statistical language model probabilities.
Our novel approach is to formulate language models as weighted sum functions on
occurrence counts.
As these weighted sum functions are monotone, we can then employ various
top-$k$ joining techniques, to efficiently find those arguments that maximize
the noisy channel's probability.
This approach can be used for all noisy channel type queries, that can be
expressed as monotone scoring functions.
We will present our method by implementing next word prediction with it.

In \cref{ch:relatedwork} an overview of related work will be given.
\Cref{ch:review} will review the two types of language models, that are used in
this thesis: \emph{Modified Kneser-Ney Smoothing} and the \emph{Generalized
Language Model}.
We will then express these language models as weighted sums in
\cref{ch:weightedsum}.
Using this representation as monotone scoring functions, top-$k$ joining
techniques and their implementations will be discussed in \cref{ch:topkjoin}.
In \cref{ch:evaluation} the given approach will be evaluated using empirical
results.
\Cref{ch:conclusion} concludes.
