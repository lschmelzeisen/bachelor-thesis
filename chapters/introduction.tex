\chapter{Introduction}
\label{ch:introduction}

\emph{Word prediction} or autocompletion is the art of inferring the next
word (or words) a user intents to type from the previous words they have
entered.
The user can either select a completion, or in the case that it did not
match, type the next indented character, with which more accurate completions
can be given.

In the context of \emph{natural language processing} the prediction requires
understanding of the user's language.
A successful general purpose word prediction will have knowledge about a
language's grammar, word frequencies, idioms, and much more.
This is in contrast to simpler solutions, like selecting the best matching word
from a list of previously entered ones in a word processing software.

\todo{For more benefits and citations read \url{http://trace.wisc.edu/docs/wordprediction2001/}.}
Word prediction has many benefits that become evident by its recent adoption
in many smartphone keyboards\footnote{For example the SwiftKey
(\url{http://swiftkey.com}) or Swype (\url{http://www.swype.com}) keyboard
applications (both visited on Aug.~1,~2015).}.
Its most obvious utilization is to speed up text entry, where it is most
beneficial to slow typists, or typists on a slow medium \parencite{Trnka2007}.
Furthermore word prediction has been shown to reduce spelling or grammar
mistakes for users with spelling disorders \parencite{Newell1992}.
But it may also be of used as a method to start computing answers to a user
query that they have not even finished typing \noref.

Nowadays word prediction is a part of various software systems.
Although word prediction has been primarily used to augment the communication of
users with a difficulty to express themselves verbally
\parencite{Swiffin1987,Trnka2011}, the benefits of a strong word prediction
should be obvious to anyone typing large amounts of text on a computer:
be it the list of possible queries, offered by popular search engines such as
Google, or the next word suggestions that many modern mobile phones provide, to
speed up the slow two-thumbs typing.

This thesis will address the problem of \emph{next} word prediction.
Our point of view is that we predict only one following word, although the
techniques described in this thesis can also be applied to the prediction
of whole sequences.
Good interfaces usually suggest a list of next word predictions and allow the
user to select from these.
This is done to mitigate the error, should the top prediction turn out not to
match the user's intend.

We formalize the task of next word prediction as follows:
\begin{equation}
  \label{eq:prefixquery}
  \NWP[p][k]{h} =
    \Argmax*[k]{\mathclap{\substack{w \in \Vocab,\\p \: \text{prefix of} \: w}}}
      \Prob{w}{h}
\end{equation}
Our task is such: given a history $h$ of entered words and an already entered
prefix $p$ of the intended word, we predict $k$ completions.
Of course $p$ may also be empty which corresponds to the case that the user has
not yet typed any prefix.
We subsume next word prediction with or without a prefix under the title of
\emph{prefix queries}, with a possibly empty prefix.
We solve this task by finding the $k$ words $w$ in the vocabulary $\Vocab$, the
set of all words, that maximize the probability $\Prob{w}{h}$.
That probability models the likelihood of a word $w$ following the history $h$.
$\Argmax*[k]$ is a helper used to model the $k$ words that maximize the
probability and an extension to the conventional $\Argmax$ function:
\begin{subequations}
  \begin{align}
    \Argmax [1]{x} f(x) &= \big\{ x \,\Given\, \forall y : f(y) \leq f(x) \big\} \\
    \Argmax [k]{x} f(x) &= \big\{ x \,\Given\, \forall y \notin \Argmax*[k-1]{z} f(z) : f(y) \leq f(x) \big\} \\
    \Argmax*[k]{x} f(x) &= \bigcup_{{1 \leq i \leq k}} \Argmax[i]{x} f(x)
  \end{align}
\end{subequations}

The quality of predictions directly correlates with the aptitude of the chosen
probability measure for $\Prob{w}{h}$.
Finding good ways to estimate these probabilities is the art of
\emph{statistical language models} (\cref{sec:introduction-langmodels}).

Computing next word predictions following \cref{eq:prefixquery} is a very time
expensive operation.
Depending on the chosen probability estimate more or less complex calculations
have to be performed, which usually require access to backing data structures.
Furthermore that probability is calculated for all words in the vocabulary
$\Vocab$, or at least for the set of all words one wants to use for prediction.

Multi-word prediction is typically implemented using \emph{Viterbi-} or
\emph{beam search}-algorithms \parencite{Bickel2005,JurafskyMartin2009}.
These algorithms are based on following most promising prediction paths and
pruning the search space with some heuristic.
They however are not applicable to the problem of predicting the next word only
and at least \textcite{Bickel2005} still employ the na\"{\i}ve method of
querying all known words to find the next word in a multi-word prediction.

This thesis concerns itself with reducing the necessary query time of prefix
queries.
Fast word prediction is desirable because to help a user reduce keystrokes,
predictions must be available to the user as soon as each keystroke is
performed.
But the ability to calculate word prediction in a manner faster than the reaction
time of a human user might also important:
for example in a client-server setting, where a central server has to calculate
word prediction queries for a number of clients, the efficiency of the employed
algorithm directly determines the necessary hardware of that server.

To this end two independent optimizations will be presented:
\begin{enumerate}
  \item We will show a novel way to rewrite the recursive definitions for
    $\Prob{w}{h}$ of the two state-of-the-art language modeling techniques
    \emph{Modified Kneser-Ney Smoothing} and the \emph{Generalized Language
    Model} as weighted sums.
    In this representation the com\-pu\-ta\-tion-costly weights will be
    independent of $w$ and can be calculated once in a precomputation step for
    each query.
    This will significantly reduce the computation time of $\Prob{w}{h}$ for
    varying words~$w$ and a fixed history~$h$.
  \item We will apply well known \emph{top-$k$ joining techniques}
    (\cref{sec:introduction-topkjoin}) to the problem of next word prediction.
    By utilizing a sophisticated data structure that allows retrieval of
    sequence-count-lists sorted by counts for arbitrary sequence prefixes,
    we will be able to considerably reduce execution time of next word
    prediction by avoiding to enumerate all words in the vocabulary.
\end{enumerate}

These optimizations will allow us to give fast, \emph{non-approximative}
predictions.
That is the predictions will be as good as the underlying language model allows:
the system will always predict exactly these words which maximize the
probability.
No pruning of the word distribution's long tail is necessary.

The remainder of this chapter will introduce the two main subjects of this
thesis: statistical language models (\cref{sec:introduction-langmodels}) and
top-$k$ join queries (\cref{sec:introduction-topkjoin}).



% ------------------------------------------------------------------------------
\section{Statistical Language Models}
\label{sec:introduction-langmodels}

\emph{Statistical language models} assign a probability $\Prob[\text{LM}]{s}$ to
a sequence of words $s = w_1^l = w_1 \ldots w_l$ by means of a probability
distribution, created through statistical analysis of text corpora.
Through their use we model knowledge of the user's language in the word
prediction application.

A major problem in creating statistical language models from text corpora, is
that even for very large corpora, most possible word sequences in a language
will not be observed in training \noref.
A common approximation is to make a $n$th-order \emph{Markov assumption},
i.e.\ to assume that the probability of the word to be predicted $w_l$ only
depends on the $n\!-\!1$~previous words $w_{l-n+1}^{l-1}$ rather than the full
sequence $w_1^{l-1}$.
Thus only sequences of length $n$ have to be considered.
The resulting language models are called \emph{$n$-gram models}.
State-of-the-art language modeling uses $n$-grams with $n$ as large as $5$
\parencite{JurafskyMartin2009,Goodman2001,Stolcke2000}.

Most language model probability formulas are given in recursive form, that only
implicitly depend on the occurrence count of sequences in the corpus \noref.
If these probabilities can be expressed as monotone scoring functions of
occurrence counts, we can apply top-$k$ joining techniques in order to solve
noisy channel type queries.


% ------------------------------------------------------------------------------
\section{Top-\emph{k} Join Queries}
\label{sec:introduction-topkjoin}

Top-$k$ queries are queries whose results are limited to only the $k$ most
important ones.
These top-$k$ results are identified by scoring all answers with a
\emph{scoring function}, and selecting the $k$ highest scoring.
\emph{Top-$k$ join queries} are these top-$k$ queries that produce
their results trough combination of multiple data sources \parencite{Ilyas2008}.

One example might be the query to a web search engine.
Search results are ranked by relevance, as computed by several criteria
(matching keywords, links from other sites, popularity, \ldots).
Users generally require more than just the one highest ranking search result, as
it may not be exactly what they were looking for.
On the other hand it would be infeasible to relevance-rank all sites in the
search engine's database.
Many search engines solve this by only displaying the top 10 sites, deemed most
relevant to the user's query, and querying more if required.

\begin{draft}
Top-$k$ joining techniques have widespread use and are of importance in
many applications \noref.
\end{draft}

As it shows in the example, the na\"{\i}ve way to solve top-$k$ join
queries, i.e.\ to aggregate all possible join results, sort all of them by
score, and then to discard non top-$k$ results, is in most cases far to
slow to be of any use.
There are of plethora of different top-$k$ processing techniques, but most
of them achieve efficiency by requiring a \emph{monotone scoring function} and
then sorting data sources on each predicate of that function
\parencite{Ilyas2008}.


% ------------------------------------------------------------------------------
\section{Thesis Overview}

This thesis is structured as follows:

\Cref{ch:review} will introduce the used notation and review the considered
language models in this thesis: \emph{Modified Kneser-Ney Smoothing} and the
\emph{Generalized Language Model}.

In \cref{ch:weightedsum} will describe a novel approach of how the traditional,
recursive definitions of these language models can be rewritten as weighted
sums.
Additionally we will show that these language models are actually monotone
scoring functions over occurrence counts.

Using this representation, \cref{ch:topkjoin} will discuss how top-$k$ joining
techniques can be applied to the problem of next word prediction.
Two widely used techniques will be analyzed: the \emph{Threshold Algorithm} and
the {No Random Access Algorithm}.

In \cref{ch:evaluation} we will evaluate the findings of this thesis and
research how well they perform in practice.

Finally, \cref{ch:conclusion} will conclude the thesis and discuss future work.
