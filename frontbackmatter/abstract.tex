\pdfbookmark[1]{Abstract}{Abstract}
\begin{abstractpage}
  \begin{abstract}
    Next word prediction is the task of guessing the next word a user intends to
    type from the words they have already entered.
    Traditionally, this problem is solved by calculating an argmax of language
    model probabilities for all words in a vocabulary.
    However, this approach is slow and becomes linearly worse with
    increasing vocabulary size.
    This thesis proposes two independent optimizations.
    First, a novel approach is presented that allows for performing part of the
    probability calculation in a precomputation step.
    Second, it is shown how to apply top-k joining techniques to word
    prediction to avoid iterating all words in the vocabulary.
    Using both optimizations sub-millisecond next word prediction time is
    achieved.
  \end{abstract}

  \selectlanguage{ngerman}
  \begin{abstract}
    Next-Word-Prediction ist die Problemstellung das nächste Wort vorherzusagen,
    welches ein Benutzer zu tippen beabsichtigt, mit Hilfe der Kenntnis, welche
    Wörter bereits eingetippt wurden.
    Für gewöhnlich wird dieses Problem mit Hilfe eines Argmax über
    Sprachmodellwahrscheinlichkeiten aller Wörter eines Vokabulars gelöst.
    Diese Ansatz ist jedoch langsam und wird mit wachsender Vokabulargröße
    linear schlechter.
    Diese Arbeit schlägt zwei unabhängige Optimierungen vor.
    Zum einen wird ein neuartiger Ansatz vorgestellt, der es erlaubt einen Teil
    der Wahrscheinlichkeitsberechnung in einem Vorberechnungsschritt
    auszuführen.
    Zum anderen wird gezeigt wie Top-k-Joining-Verfahren zur Berechnung der
    Wortvorhersage eingesetzt werden können.
    Dies ermöglicht es, die Iteration über aller Wörter des Vokabulars zu
    vermeiden.
    Unter Einsatz beider Optimierungen wurden Vorhersagezeiten unter einer
    Millisekunde erzielt.
  \end{abstract}
  \selectlanguage{american}
\end{abstractpage}

