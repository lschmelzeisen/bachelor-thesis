\documentclass[11pt,a4paper]{report}

\usepackage[utf8]{inputenc}
\usepackage[german,english]{babel}
\usepackage{csquotes}

\usepackage{xparse}

\usepackage{amsmath}
\usepackage{mathtools}
\usepackage{nicefrac}
\usepackage{IEEEtrantools}

\usepackage[parfill]{parskip}
\usepackage{appendix}

% In-paragrap enumerations
\usepackage{paralist}

% Code listings
\usepackage{listings}

% Courier as monospace font.
\usepackage{courier}

\usepackage[backend=biber,style=authoryear]{biblatex}
\addbibresource{langmodels.bib}

\usepackage{color}

% Should be included last, to redefine commands
\usepackage{hyperref}

% ------------------------------------------------------------------------------

\newcommand{\mytitle}{Computing Language Model Arg-Max-Queries Using Top-\emph{k}
  Joining Techniques}
%\newcommand{\mytitle}{Applying Top-\emph{k} Joining Techniques to
%  Modified-Kneser-Ney Language Model's and Generalized Language Model's Argmax
%  Queries}
\newcommand{\myname}{Lukas Schmelzeisen}
\newcommand{\mymail}{lukas@uni-koblenz.de}
\newcommand{\myaddress}{University of Koblenz-Landau, Germany}

% -- Math ----------------------------------------------------------------------

% given
% see http://tex.stackexchange.com/questions/141570/sizing-for-given-that-symbol-vertical-bar
\newcommand\givenbase[1][]{\,#1\lvert\,}
\let\given\givenbase
\newcommand\sgiven{\givenbase[\delimsize]}
\DeclarePairedDelimiterX\Basics[1](){\let\given\sgiven #1}

% set
% see http://tex.stackexchange.com/questions/13634/define-pretty-sets-in-latex-esp-how-to-do-the-condition-separator
\newcommand\SetSymbol[1][]{
  \nonscript\,#1\vert\nonscript\,\mathopen{}\allowbreak
}
\DeclarePairedDelimiterX\Set[1]{\lbrace}{\rbrace}%
  { \renewcommand\given{\SetSymbol[\delimsize]} #1 }

% argmax
\DeclareMathOperator*{\argmax}{arg\,max}

% cardinality
\newcommand{\cardinality}[1]{|#1|}

% probability
\NewDocumentCommand{\prob}{s O{} g g}{
  \IfBooleanT{#1}{\hat}P_{#2}
  \IfNoValueTF{#3}{}{(\IfNoValueTF{#4}{#3}{#3 \given #4})}
}

% next word prediction
\newcommand{\nwp}[1]{\operatorname{NWP}(#1)}

% next keystroke savings
\newcommand{\nksSymbol}{\operatorname{NKS}}
\newcommand{\nks}[1]{\nksSymbol(#1)}

% -- Code ----------------------------------------------------------------------

\newcommand{\inlinecode}[1]{\lstinline[columns=fixed]{#1}}

% -- Editiorial Commands -------------------------------------------------------

\newcommand{\todo}[1]{\textcolor{red}{\textbf{TODO}: #1}}
\newcommand{\lukas}[1]{\textcolor{cyan}{Lukas: #1}}
\newcommand{\rene}[1]{\textcolor{blue}{Rene: #1}}

\newcommand{\noref}{\textcolor{red}{(NoRef)}}
\newcommand{\mbref}[1]{\textcolor{magenta}{(#1)}}

% ------------------------------------------------------------------------------

\title{\mytitle}
\author{
  \myname \\
  \small{\myaddress} \\
  \small{\texttt{\href{mailto:\mymail}{\nolinkurl{\mymail}}}}
}
\date{\today}
\hypersetup{
  unicode = true,
  pdfauthor = \myname,
  pdftitle = \mytitle,
}

\begin{document}

% ==============================================================================
\maketitle

% ==============================================================================
\vspace*{\fill}
\begin{minipage}{\linewidth} % To have both abstracts on one page
  \begin{abstract}
    Abstract.
  \end{abstract}

  \vspace{5em}

  \selectlanguage{german}
  \begin{abstract}
    Diesmal auf deutsch.
  \end{abstract}
  \selectlanguage{english}
\end{minipage}
\vspace*{\fill}

% ==============================================================================
\tableofcontents

% ==============================================================================
\chapter{Introduction}

A statistical \emph{language model} assigns a probability to a sequence of
words.

Language models are important and used in practice.

State-of-the-art language models are \emph{Modified-Kneser-Ney Language Model}
and \emph{Generalized Language Model}

Modified-Kneser-Ney Language Model is:
\begin{equation}
  \prob[\text{MKN}]{\ldots} = \ldots
\end{equation}

Genralized Language Model is \parencite{Pickhardt2014}:
\begin{equation}
  \prob[\text{GLM}]{\ldots} = \ldots
\end{equation}

We want to efficiently compute argmax.

We map argmax onto researched problem of top-k joining.

We apply that technique.

Most non-trivial top-\emph{k} joining algorithms require a monotone scoring
function, some further improvements can be made if using a linear scoring
function~\parencite{Ilyas2008}.

% ==============================================================================
\chapter{Related Work}

% ==============================================================================
\chapter{Language Models}

% ==============================================================================
\chapter{Formulating Language Models as weighted sums}
\label{chp-lm-score-func}

% ------------------------------------------------------------------------------
\section{Modified-Kneser-Ney Language Model}

% ------------------------------------------------------------------------------
\section{Generalized Language Model}

Binomial Diamond.

\subsection{Bitmagic}

% ==============================================================================
\chapter{Top-\emph{k} Joining Language Models}

\lstset{
  language=SQL,
  basicstyle=\ttfamily,
  emph={arg,Pattern,Pattern1,PatternN,P_LM},
  emphstyle=\textit,
  deletekeywords={count} % Used as column name, don't want to escape it though.
}

This chapter explains how to efficiently compute arg-max-queries of language
model $\prob[\text{LM}]$ using top-$k$ joining techniques.

The task is to efficiently compute the following expression:
\begin{equation}
  \argmax_{t \in T} \prob[\text{LM}]{t}{h}
\end{equation}
Where $t$ is a token in the set of all tokens $T$, $\prob[\text{LM}]$ is a
language model, that can be expressed as a weighted sum (as described in
chapter~\ref{chp-lm-score-func}), and $h$ is the conditional history, a sequence
of tokens.

Given a language model $\prob[\text{LM}]$ as a scoring function, if we
imagine our input data forming database relations, we can express our
arg-max-query as a SQL-\texttt{SELECT}-Request.

\begin{lstlisting}[mathescape]
  SELECT
    Token.token AS arg,
    $\prob[\text{LM}]$(Pattern1.count, ..., PatternN.count) AS score
  FROM Token
  LEFT OUTER JOIN Pattern1
    ON history(Pattern1) + arg = Pattern1.sequence
  ...
  LEFT OUTER JOIN PatternN
    ON history(PatternN) + arg = PatternN.sequence
  ORDER BY score DESC
  LIMIT k;
\end{lstlisting}

\todo{Where \inlinecode{history} and plus is concat.}
\todo{Explain \inlinecode{LEFT OUTER JOIN}.}

Our data model is thus: One input relation \inlinecode{Token} that stores
all possible tokens \inlinecode{arg}, over which we want to compute:
\[
\argmax_{\text{\inlinecode{arg}} \, \in \, \text{\inlinecode{Token}}}
  \prob[\text{LM}]{\text{\inlinecode{arg}}}{\text{\inlinecode{history}}}
\]
Additionally we have $n$ input relations
\inlinecode{Pattern1} to \inlinecode{PatternN} that contain pairs
of \inlinecode{sequence}s and the sequence's occurrence \inlinecode{count} as
columns.
Each \inlinecode{Pattern} relation contains exactly these sequences that fit the
relations pattern.

For an example database state and fully expanded SQL-Request see
appendix~\ref{app-sql-example}.

A naïve method to solve this query would
\begin{inparaenum}[(1)]
  \item produce all possible object combinations that satisfy the join
    condition,
  \item order results by scoring function $\prob[\text{LM}]$, and
  \item discard non top-$k$ objects (those that do not have the $k$ largest
    scores).
\end{inparaenum}

Considering that in reality only small values of $k$ are queried, this approach
can be deemed as rather wasteful, as large numbers of objects would have to be
discarded.
Optimally we would like to cherry-pick the top-$k$ join results and only
score and order those.
Relying on the fact that we are dealing with a \emph{monotone} scoring function
we approximate this by sorting our input relations by their counts.
\cite{Ilyas2008} survey a wide number of techniques to produces top-$k$ results
with a monotone scoring function from sorted relations.

Among other things, top-$k$ joining techniques differ on the type of join
conditions they can be applied to.
One particularly optimizable join operation is the \emph{equi-join}:
the join that checks for equality over a unique key attribute present in all
relations.
Equi-joins have the advantage that it is possible to form equivalence classes
over an object's key attribute to determine which join results an object is
part of.
Using this, the \emph{hash join} algorithm can efficiently generate join
results.

It is possible to view our join operations as an instance of equi-join:
By filtering each input relation \inlinecode{Pattern} to only contain
objects whose sequences begin with the requested history, objects'
sequences only differ on the last token (our argument token \inlinecode{arg}).
Our join condition thus reduces to
\inlinecode{arg = lastToken(Pattern.sequence)}.

\todo{Read \textit{``2013 BlanasPatel Memory Footprint Matters Efficient
Equi-Join Algorithms fro Main Memory Data Processing.pdf''} for possibly better
join algorithm}.

The technique for obtaining these both sorted and filtered relations is
explained in section~\ref{sec-sorted-and-filtered-access}.

\todo{Do we want discussion here rather than explanation?}

The algorithm used to produce top-$k$ results is presented in
section~\ref{sec-algorithm}.

% ------------------------------------------------------------------------------
\section{Remarks}

\todo{}.

% ------------------------------------------------------------------------------
\section{Sorted and filtered access}
\label{sec-sorted-and-filtered-access}

As explained we require a data structure for efficiently retrieving all
sequences whose first to second to last words match a requested string $h$.
Additionally we want to specify that the last word has to contain $p$ as a
prefix, in order to support efficient Next-Keystroke-Savings $\nksSymbol$
computation.
\mbox{$\Set{\, w_1^n \: \given \: w_1^{n-1} = h \; \land \; b \:\text{is prefix of}\: w_n \,}$}
which is equivalent to
\mbox{$\Set{\, w_1^n \: \given \: \,}$}
\begin{equation*}
  \underbracket{\enskip w_1 \qquad w_2 \qquad \ldots \qquad w_{n-1} \enskip}
    _\text{matches $h$}
  \quad
  \underbracket{\enskip w_n \enskip}_\text{starts with $b$}
\end{equation*}


NKSS requires Top-\emph{k} Completion.

Use Completion-Trie \parencite{HsuOttaviano2013}.

\todo{Explain Completion-Trie here?}
\todo{Can we save all patterns in one trie?}

% ------------------------------------------------------------------------------
\section{Algorithm}
\label{sec-algorithm}

\subsection{Requirements}

\begin{enumerate}
  \item Dataset remains the same.
  \item Scoring function changes on each query.
  \item Join condition changes on each query!
\end{enumerate}

\subsection{Algorithm}

There are several techniques for indexing {Rank Join Indices \mbref{Tsaparas et
al 2003}, Onion Indices \mbref{Chang et al 2000}) which we don't use as history
changes to often.

Ranked Join Indicies\mbref{Tsaparas et al 2003})

Algorithms that precompute stuff may be general speed up, but not in our case
since we never expect the same thing to be queried again.

Rank-Join \parencite{Ilyas2004}.

Evaluate following algorithms:

\begin{enumerate}
  \item Traditional Rank-Join that generates all possible join combinations
    after each input.
  \item Improved version that uses random access for early tuple completion.
\end{enumerate}

Use observation that if a pattern is seen, it's child patterns' will be seen as
well.

Clever: How to find T as max of $(p_1^max, p_2^last)$ and $(p_1^last, p_2^max)$.

Clever: Priority-Queue with constant contains and update.

% ==============================================================================
\chapter{Evaluation}

No comparision with existing auto completion engines, as the goal was to solve
language model argmax, not fast auto completion.

% ==============================================================================
\chapter{Conclusion}

% ==============================================================================
\printbibliography

\begin{appendices}

% ==============================================================================
\chapter{Example SQL-Request}
\label{app-sql-example}

\todo{With database examples, and fully expanded SQL code, probably MKN query.}

% ==============================================================================
\chapter{Notes}

\todo{Remove before releasing.}

% ------------------------------------------------------------------------------
\section{SQL Query Expansion}

\lstset{
  language=SQL,
  emph={token,pattern,pattern1,patternN},
  emphstyle=\textit,
  deletekeywords={count} % Used as column name, don't want to escape it though.
}

\subsection{Modified Kneser Ney Language Model}

\begin{lstlisting}
  SELECT
    token.token AS arg,
    func( 1.count,  11.count, 111.count,
         x1.count, x11.count)
      AS score
  FROM  token
  LEFT OUTER JOIN    1
    ON history(  1) + token =   1.sequence
  LEFT OUTER JOIN   11
    ON history( 11) + token =  11.sequence
  LEFT OUTER JOIN  111
    ON history(111) + token = 111.sequence
  LEFT OUTER JOIN   x1
    ON history( x1) + token =  x1.sequence
  LEFT OUTER JOIN  x11
    ON history(x11) + token = x11.sequence
  ORDER BY score DESC;
  LIMIT k;
\end{lstlisting}

\clearpage
\subsection{Generalized Language Model}

\begin{lstlisting}
  SELECT
    token.token AS arg,
    func( 001.count,  011.count,  101.count, 111.count,
         x001.count, x011.count, x101.count)
      AS score
  SELECT token, score
  FROM token
  LEFT OUTER JOIN  001
    ON history( 001) + token =  001.sequence
  LEFT OUTER JOIN  011
    ON history( 011) + token =  011.sequence
  LEFT OUTER JOIN  101
    ON history( 101) + token =  101.sequence
  LEFT OUTER JOIN  111
    ON history( 111) + token =  111.sequence
  LEFT OUTER JOIN x001
    ON history(x001) + token = x001.sequence
  LEFT OUTER JOIN x011
    ON history(x011) + token = x011.sequence
  LEFT OUTER JOIN x101
    ON history(x101) + token = x101.sequence
  ORDER BY score DESC;
  LIMIT k;
\end{lstlisting}

\end{appendices}

\end{document}
